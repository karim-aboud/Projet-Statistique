% Options for packages loaded elsewhere
\PassOptionsToPackage{unicode}{hyperref}
\PassOptionsToPackage{hyphens}{url}
%
\documentclass[
]{article}
\usepackage{amsmath,amssymb}
\usepackage{lmodern}
\usepackage{iftex}
\ifPDFTeX
  \usepackage[T1]{fontenc}
  \usepackage[utf8]{inputenc}
  \usepackage{textcomp} % provide euro and other symbols
\else % if luatex or xetex
  \usepackage{unicode-math}
  \defaultfontfeatures{Scale=MatchLowercase}
  \defaultfontfeatures[\rmfamily]{Ligatures=TeX,Scale=1}
\fi
% Use upquote if available, for straight quotes in verbatim environments
\IfFileExists{upquote.sty}{\usepackage{upquote}}{}
\IfFileExists{microtype.sty}{% use microtype if available
  \usepackage[]{microtype}
  \UseMicrotypeSet[protrusion]{basicmath} % disable protrusion for tt fonts
}{}
\makeatletter
\@ifundefined{KOMAClassName}{% if non-KOMA class
  \IfFileExists{parskip.sty}{%
    \usepackage{parskip}
  }{% else
    \setlength{\parindent}{0pt}
    \setlength{\parskip}{6pt plus 2pt minus 1pt}}
}{% if KOMA class
  \KOMAoptions{parskip=half}}
\makeatother
\usepackage{xcolor}
\IfFileExists{xurl.sty}{\usepackage{xurl}}{} % add URL line breaks if available
\IfFileExists{bookmark.sty}{\usepackage{bookmark}}{\usepackage{hyperref}}
\hypersetup{
  pdftitle={Projet - Maladies cardiaques},
  hidelinks,
  pdfcreator={LaTeX via pandoc}}
\urlstyle{same} % disable monospaced font for URLs
\usepackage[margin=1in]{geometry}
\usepackage{color}
\usepackage{fancyvrb}
\newcommand{\VerbBar}{|}
\newcommand{\VERB}{\Verb[commandchars=\\\{\}]}
\DefineVerbatimEnvironment{Highlighting}{Verbatim}{commandchars=\\\{\}}
% Add ',fontsize=\small' for more characters per line
\usepackage{framed}
\definecolor{shadecolor}{RGB}{248,248,248}
\newenvironment{Shaded}{\begin{snugshade}}{\end{snugshade}}
\newcommand{\AlertTok}[1]{\textcolor[rgb]{0.94,0.16,0.16}{#1}}
\newcommand{\AnnotationTok}[1]{\textcolor[rgb]{0.56,0.35,0.01}{\textbf{\textit{#1}}}}
\newcommand{\AttributeTok}[1]{\textcolor[rgb]{0.77,0.63,0.00}{#1}}
\newcommand{\BaseNTok}[1]{\textcolor[rgb]{0.00,0.00,0.81}{#1}}
\newcommand{\BuiltInTok}[1]{#1}
\newcommand{\CharTok}[1]{\textcolor[rgb]{0.31,0.60,0.02}{#1}}
\newcommand{\CommentTok}[1]{\textcolor[rgb]{0.56,0.35,0.01}{\textit{#1}}}
\newcommand{\CommentVarTok}[1]{\textcolor[rgb]{0.56,0.35,0.01}{\textbf{\textit{#1}}}}
\newcommand{\ConstantTok}[1]{\textcolor[rgb]{0.00,0.00,0.00}{#1}}
\newcommand{\ControlFlowTok}[1]{\textcolor[rgb]{0.13,0.29,0.53}{\textbf{#1}}}
\newcommand{\DataTypeTok}[1]{\textcolor[rgb]{0.13,0.29,0.53}{#1}}
\newcommand{\DecValTok}[1]{\textcolor[rgb]{0.00,0.00,0.81}{#1}}
\newcommand{\DocumentationTok}[1]{\textcolor[rgb]{0.56,0.35,0.01}{\textbf{\textit{#1}}}}
\newcommand{\ErrorTok}[1]{\textcolor[rgb]{0.64,0.00,0.00}{\textbf{#1}}}
\newcommand{\ExtensionTok}[1]{#1}
\newcommand{\FloatTok}[1]{\textcolor[rgb]{0.00,0.00,0.81}{#1}}
\newcommand{\FunctionTok}[1]{\textcolor[rgb]{0.00,0.00,0.00}{#1}}
\newcommand{\ImportTok}[1]{#1}
\newcommand{\InformationTok}[1]{\textcolor[rgb]{0.56,0.35,0.01}{\textbf{\textit{#1}}}}
\newcommand{\KeywordTok}[1]{\textcolor[rgb]{0.13,0.29,0.53}{\textbf{#1}}}
\newcommand{\NormalTok}[1]{#1}
\newcommand{\OperatorTok}[1]{\textcolor[rgb]{0.81,0.36,0.00}{\textbf{#1}}}
\newcommand{\OtherTok}[1]{\textcolor[rgb]{0.56,0.35,0.01}{#1}}
\newcommand{\PreprocessorTok}[1]{\textcolor[rgb]{0.56,0.35,0.01}{\textit{#1}}}
\newcommand{\RegionMarkerTok}[1]{#1}
\newcommand{\SpecialCharTok}[1]{\textcolor[rgb]{0.00,0.00,0.00}{#1}}
\newcommand{\SpecialStringTok}[1]{\textcolor[rgb]{0.31,0.60,0.02}{#1}}
\newcommand{\StringTok}[1]{\textcolor[rgb]{0.31,0.60,0.02}{#1}}
\newcommand{\VariableTok}[1]{\textcolor[rgb]{0.00,0.00,0.00}{#1}}
\newcommand{\VerbatimStringTok}[1]{\textcolor[rgb]{0.31,0.60,0.02}{#1}}
\newcommand{\WarningTok}[1]{\textcolor[rgb]{0.56,0.35,0.01}{\textbf{\textit{#1}}}}
\usepackage{graphicx}
\makeatletter
\def\maxwidth{\ifdim\Gin@nat@width>\linewidth\linewidth\else\Gin@nat@width\fi}
\def\maxheight{\ifdim\Gin@nat@height>\textheight\textheight\else\Gin@nat@height\fi}
\makeatother
% Scale images if necessary, so that they will not overflow the page
% margins by default, and it is still possible to overwrite the defaults
% using explicit options in \includegraphics[width, height, ...]{}
\setkeys{Gin}{width=\maxwidth,height=\maxheight,keepaspectratio}
% Set default figure placement to htbp
\makeatletter
\def\fps@figure{htbp}
\makeatother
\setlength{\emergencystretch}{3em} % prevent overfull lines
\providecommand{\tightlist}{%
  \setlength{\itemsep}{0pt}\setlength{\parskip}{0pt}}
\setcounter{secnumdepth}{-\maxdimen} % remove section numbering
\ifLuaTeX
  \usepackage{selnolig}  % disable illegal ligatures
\fi

\title{Projet - Maladies cardiaques}
\author{}
\date{\vspace{-2.5em}2022-06-21}

\begin{document}
\maketitle

\hypertarget{introduction}{%
\section{Introduction}\label{introduction}}

Les maladies cardiovasculaires (MCV) sont la première cause de décès
dans le monde. On estime que 17,9 millions de personnes en meurent
chaque année, ce qui représente 31 \% de tous les décès dans le monde.
Quatre décès par MCV sur cinq sont dus à des crises cardiaques et à des
accidents vasculaires cérébraux, et un tiers de ces décès surviennent
prématurément chez les personnes de moins de 70 ans. L'insuffisance
cardiaque est un événement courant causé par les MCV et cet ensemble de
données contient 11 caractéristiques qui peuvent être utilisées pour
prédire une éventuelle maladie cardiaque.

Les personnes atteintes d'une maladie cardiovasculaire ou présentant un
risque cardiovasculaire élevé (en raison de la présence d'un ou
plusieurs facteurs de risque tels que l'hypertension, le diabète,
l'hyperlipidémie ou une maladie déjà établie) ont besoin d'une détection
et d'une prise en charge précoces, pour lesquelles un modèle
d'apprentissage automatique peut être d'une grande aide.

\hypertarget{uxe0-propos-des-donnuxe9es}{%
\subsection{À propos des données}\label{uxe0-propos-des-donnuxe9es}}

Ci dessous les different variables présentes dans notre jeu de données

\begin{enumerate}
\def\labelenumi{\arabic{enumi}.}
\item
  \textbf{Age} : âge du patient {[}années{]}.
\item
  \textbf{Sex} : sexe du patient {[}M : Male, F : Female{]}.
\item
  \textbf{ChestPainType} : type de douleur thoracique {[}TA : Angine
  typique, ATA : Angine atypique, NAP: Douleur non angineuse, ASY :
  Asymptomatique{]}.
\item
  \textbf{RestingBP} : pression artérielle au repos {[}mm Hg{]}.
\item
  \textbf{Cholestérol} : cholestérol sérique {[}mm/dl{]}.
\item
  \textbf{FastingBS} : glycémie à jeun {[}1 : si FastingBS
  \textgreater{} 120 mg/dl, 0 : sinon{]}.
\item
  \textbf{RestingECG} : résultats de l'électrocardiogramme au repos
  {[}Normal : normal, ST : présentant une anomalie de l'onde ST-T
  (inversions de l'onde T et/ou élévation ou dépression du segment ST de
  \textgreater{} 0,05 mV), HVG : présentant une hypertrophie
  ventriculaire gauche probable ou certaine selon les critères
  d'Estes{]}.
\item
  \textbf{MaxHR} : fréquence cardiaque maximale atteinte {[}Valeur
  numérique comprise entre 60 et 202{]}.
\item
  \textbf{ExerciseAngina} : angine de poitrine induite par l'exercice
  {[}Y : Oui, N : Non{]}.
\item
  \textbf{Oldpeak} : oldpeak = ST: Dépression du segment ST. La
  dépression ST désigne une constatation sur un électrocardiogramme,
  dans laquelle le tracé du segment ST est anormalement bas par rapport
  à la ligne de base.
\item
  \textbf{ST\_Slope} : la pente du segment ST du pic d'exercice {[}Up :
  en pente ascendante, Flat : plat, Down : en pente descendante{]}.
\item
  \textbf{HeartDisease} : classe de sortie {[}1 : maladie cardiaque, 0 :
  normal{]}.
\end{enumerate}

\hypertarget{import-des-packages}{%
\paragraph{Import des packages}\label{import-des-packages}}

\begin{Shaded}
\begin{Highlighting}[]
\FunctionTok{library}\NormalTok{(ggplot2)}
\FunctionTok{library}\NormalTok{(dplyr)}
\end{Highlighting}
\end{Shaded}

\hypertarget{lecture-du-jeu-de-donnuxe9es}{%
\paragraph{Lecture du jeu de
données}\label{lecture-du-jeu-de-donnuxe9es}}

\begin{Shaded}
\begin{Highlighting}[]
\NormalTok{heart }\OtherTok{\textless{}{-}} \FunctionTok{read.csv}\NormalTok{(}\StringTok{"heart.csv"}\NormalTok{)}
\FunctionTok{attach}\NormalTok{(heart)}
\end{Highlighting}
\end{Shaded}

\hypertarget{analyse-de-la-structure-des-donnuxe9es}{%
\paragraph{Analyse de la structure des
données}\label{analyse-de-la-structure-des-donnuxe9es}}

\begin{Shaded}
\begin{Highlighting}[]
\FunctionTok{str}\NormalTok{(heart)}
\end{Highlighting}
\end{Shaded}

\begin{verbatim}
## 'data.frame':    918 obs. of  12 variables:
##  $ Age           : int  40 49 37 48 54 39 45 54 37 48 ...
##  $ Sex           : chr  "M" "F" "M" "F" ...
##  $ ChestPainType : chr  "ATA" "NAP" "ATA" "ASY" ...
##  $ RestingBP     : int  140 160 130 138 150 120 130 110 140 120 ...
##  $ Cholesterol   : int  289 180 283 214 195 339 237 208 207 284 ...
##  $ FastingBS     : int  0 0 0 0 0 0 0 0 0 0 ...
##  $ RestingECG    : chr  "Normal" "Normal" "ST" "Normal" ...
##  $ MaxHR         : int  172 156 98 108 122 170 170 142 130 120 ...
##  $ ExerciseAngina: chr  "N" "N" "N" "Y" ...
##  $ Oldpeak       : num  0 1 0 1.5 0 0 0 0 1.5 0 ...
##  $ ST_Slope      : chr  "Up" "Flat" "Up" "Flat" ...
##  $ HeartDisease  : int  0 1 0 1 0 0 0 0 1 0 ...
\end{verbatim}

On a:

\begin{itemize}
\tightlist
\item
  5 variables quantitatives continue (Age, RestingBP, Cholesterol,
  MaxHR, Oldpeak).
\item
  6 variables qualitatives nominale (Sex, ChestPainType, FastingBS,
  RestingECG, ExerciceAngina, ST\_Slopem, HeartDisease) dont 1 variable
  cible (HeartDisease)
\end{itemize}

Avant d'explorer les données, convertissons les attributs de classe en
fonction des besoins de notre analyse.

En effet, les facteurs sont des variables dans R qui prennent un nombre
limité de valeurs différentes. Ces variables sont souvent appelées
variables catégorielles.

L'une des utilisations les plus importantes des facteurs est la
modélisation statistique, puisque les variables catégoriques entrent
dans les modèles statistiques différemment des variables continues, le
stockage des données en tant que facteurs garantit que les fonctions de
modélisation traiteront ces données correctement.

En observant la structure de notre jeu de données, nous pouvons dire les
points suivants:-

\begin{itemize}
\item
  \textbf{FastingBS} ne peut pas être une variable continue car elle
  indique seulement si une personne a jeûné pendant la nuit. Nous devons
  la convertir en facteur et l'étiqueter à notre convenance.
\item
  \textbf{HeartDisease} est la variable prédictive et nous indique si
  l'individu a une insuffisance cardiaque ou non. Par conséquent, nous
  convertissons la variable en facteur et l'étiquetons à notre
  convenance.
\end{itemize}

\begin{Shaded}
\begin{Highlighting}[]
\NormalTok{heart}\SpecialCharTok{$}\NormalTok{FastingBS }\OtherTok{\textless{}{-}} \FunctionTok{as.factor}\NormalTok{(heart}\SpecialCharTok{$}\NormalTok{FastingBS)}
\FunctionTok{levels}\NormalTok{(heart}\SpecialCharTok{$}\NormalTok{FastingBS) }\OtherTok{\textless{}{-}} \FunctionTok{c}\NormalTok{(}\StringTok{"No"}\NormalTok{, }\StringTok{"Yes"}\NormalTok{)}

\NormalTok{heart}\SpecialCharTok{$}\NormalTok{HeartDisease }\OtherTok{\textless{}{-}} \FunctionTok{as.factor}\NormalTok{(heart}\SpecialCharTok{$}\NormalTok{HeartDisease)}
\FunctionTok{levels}\NormalTok{(heart}\SpecialCharTok{$}\NormalTok{HeartDisease) }\OtherTok{\textless{}{-}} \FunctionTok{c}\NormalTok{(}\StringTok{"No"}\NormalTok{, }\StringTok{"Yes"}\NormalTok{)}
\end{Highlighting}
\end{Shaded}

Vérification que les changements ci-dessus sont mis en oeuvre

\begin{Shaded}
\begin{Highlighting}[]
\FunctionTok{str}\NormalTok{(heart}\SpecialCharTok{$}\NormalTok{FastingBS)}
\FunctionTok{str}\NormalTok{(heart}\SpecialCharTok{$}\NormalTok{HeartDisease)}
\end{Highlighting}
\end{Shaded}

\begin{verbatim}
##  Factor w/ 2 levels "No","Yes": 1 1 1 1 1 1 1 1 1 1 ...
##  Factor w/ 2 levels "No","Yes": 1 2 1 2 1 1 1 1 2 1 ...
\end{verbatim}

\hypertarget{analyse-univariuxe9e}{%
\section{Analyse univariée}\label{analyse-univariuxe9e}}

\hypertarget{analyse-descriptive-de-base}{%
\subsection{Analyse descriptive de
base}\label{analyse-descriptive-de-base}}

\hypertarget{la-variable-age}{%
\paragraph{\texorpdfstring{La variable
\textbf{Age}}{La variable Age}}\label{la-variable-age}}

Voyons tout d'abord la répartition globale de la variable \textbf{Age}

\begin{Shaded}
\begin{Highlighting}[]
\NormalTok{age\_hist }\OtherTok{\textless{}{-}} \FunctionTok{ggplot}\NormalTok{(heart, }\FunctionTok{aes}\NormalTok{(}\AttributeTok{x=}\NormalTok{Age)) }\SpecialCharTok{+} 
  \FunctionTok{geom\_histogram}\NormalTok{(}\AttributeTok{binwidth=}\DecValTok{1}\NormalTok{, }\AttributeTok{color=}\StringTok{"black"}\NormalTok{, }\AttributeTok{fill=}\StringTok{"lightblue"}\NormalTok{) }\SpecialCharTok{+}
  \FunctionTok{geom\_vline}\NormalTok{(}\FunctionTok{aes}\NormalTok{(}\AttributeTok{xintercept =} \FunctionTok{mean}\NormalTok{(Age)), }\AttributeTok{color =} \StringTok{"red"}\NormalTok{, }\AttributeTok{linetype =} \StringTok{"dashed"}\NormalTok{, }\AttributeTok{size =} \DecValTok{1}\NormalTok{) }\SpecialCharTok{+}
  \FunctionTok{labs}\NormalTok{(}\AttributeTok{title=}\StringTok{"Histogramme de Age"}\NormalTok{)}

\FunctionTok{print}\NormalTok{(age\_hist)}
\end{Highlighting}
\end{Shaded}

\includegraphics{heart_files/figure-latex/unnamed-chunk-6-1.pdf}

\begin{Shaded}
\begin{Highlighting}[]
\FunctionTok{mean}\NormalTok{(Age)}
\FunctionTok{sd}\NormalTok{(Age)}
\FunctionTok{summary}\NormalTok{(Age)}
\end{Highlighting}
\end{Shaded}

\begin{verbatim}
## [1] 53.51089
## [1] 9.432617
##    Min. 1st Qu.  Median    Mean 3rd Qu.    Max. 
##   28.00   47.00   54.00   53.51   60.00   77.00
\end{verbatim}

La moyenne est de 53,5 ans, ce qui veut dire que la population étudier
est une population agé.

La médiane étant de 54 ans (très proche de la moyenne), ceci implique
qu'il n'ya pas beaucoup de valeurs abérantes (sinon la moyenne serait
différente par rapport à la médiane).

On veut s'assurer que les variables quantitatives suivent une
distribution normale.

Il existent plusieurs méthodes pour vérifier si des données suivent une
loi normale ou non:

\begin{itemize}
\tightlist
\item
  En regardant l'histogramme et la courbe de densité
\item
  En traçant le QQ-plot
\item
  En testant avec shapiro.test
\end{itemize}

\begin{Shaded}
\begin{Highlighting}[]
\FunctionTok{hist}\NormalTok{(Age, }\AttributeTok{probability =} \ConstantTok{TRUE}\NormalTok{, }\AttributeTok{main =} \StringTok{"Repartition de l\textquotesingle{}age"}\NormalTok{)}

\FunctionTok{points}\NormalTok{(}\FunctionTok{seq}\NormalTok{(}\DecValTok{0}\NormalTok{,}\DecValTok{85}\NormalTok{,}\FloatTok{0.5}\NormalTok{),}\FunctionTok{dnorm}\NormalTok{(}\FunctionTok{seq}\NormalTok{(}\DecValTok{0}\NormalTok{,}\DecValTok{85}\NormalTok{,}\FloatTok{0.5}\NormalTok{),}\FunctionTok{mean}\NormalTok{(Age),}\FunctionTok{sd}\NormalTok{(Age)), }\AttributeTok{col=}\StringTok{"red"}\NormalTok{,}\AttributeTok{type=}\StringTok{"l"}\NormalTok{)}
\end{Highlighting}
\end{Shaded}

\includegraphics{heart_files/figure-latex/unnamed-chunk-8-1.pdf}

\begin{Shaded}
\begin{Highlighting}[]
\FunctionTok{qqnorm}\NormalTok{(Age)}
\FunctionTok{abline}\NormalTok{(}\FunctionTok{mean}\NormalTok{(Age),}\FunctionTok{sd}\NormalTok{(Age),}\AttributeTok{col=}\DecValTok{2}\NormalTok{)}
\end{Highlighting}
\end{Shaded}

\includegraphics{heart_files/figure-latex/unnamed-chunk-8-2.pdf}

Ici, on voit bien que la variable age est normale car les points du
graphique Q-Q se trouve sur une ligne diagonale droite.

\hypertarget{la-variable-chetpaintype}{%
\paragraph{\texorpdfstring{La variable
\textbf{ChetPainType}}{La variable ChetPainType}}\label{la-variable-chetpaintype}}

Regardons la distribution des different types de douleur thoracique

\begin{Shaded}
\begin{Highlighting}[]
\NormalTok{chest\_pain\_bar }\OtherTok{\textless{}{-}} \FunctionTok{ggplot}\NormalTok{(heart, }\FunctionTok{aes}\NormalTok{(}\AttributeTok{x=}\NormalTok{ChestPainType, }\AttributeTok{fill=}\NormalTok{ChestPainType)) }\SpecialCharTok{+}
  \FunctionTok{geom\_bar}\NormalTok{() }\SpecialCharTok{+}
  \FunctionTok{labs}\NormalTok{(}\AttributeTok{title=}\StringTok{"Distribution des types de douleur thoracique"}\NormalTok{, }\AttributeTok{ylab=}\StringTok{"Effectif"}\NormalTok{)}

\FunctionTok{print}\NormalTok{(chest\_pain\_bar)}
\end{Highlighting}
\end{Shaded}

\includegraphics{heart_files/figure-latex/unnamed-chunk-9-1.pdf}

Et en terme de pourcentage..

\begin{Shaded}
\begin{Highlighting}[]
\NormalTok{chest\_pain\_table }\OtherTok{\textless{}{-}} \FunctionTok{table}\NormalTok{(ChestPainType)}
\NormalTok{chest\_pain\_pct}\OtherTok{\textless{}{-}}\FunctionTok{round}\NormalTok{(chest\_pain\_table}\SpecialCharTok{/}\FunctionTok{sum}\NormalTok{(chest\_pain\_table)}\SpecialCharTok{*}\DecValTok{100}\NormalTok{)}

\NormalTok{lbls1}\OtherTok{\textless{}{-}}\FunctionTok{paste}\NormalTok{(}\FunctionTok{names}\NormalTok{(chest\_pain\_table),chest\_pain\_pct)}
\NormalTok{lbls}\OtherTok{\textless{}{-}}\FunctionTok{paste}\NormalTok{(lbls1, }\StringTok{"\%"}\NormalTok{, }\AttributeTok{sep=}\StringTok{""}\NormalTok{)}

\FunctionTok{pie}\NormalTok{(chest\_pain\_table, }\AttributeTok{labels =}\NormalTok{ lbls, }
    \AttributeTok{col =} \FunctionTok{rainbow}\NormalTok{(}\FunctionTok{length}\NormalTok{(lbls)),}
    \AttributeTok{main=}\StringTok{"Pie Chart of Chest Pain"}\NormalTok{,}\AttributeTok{radius =} \FloatTok{0.9}\NormalTok{)}
\end{Highlighting}
\end{Shaded}

\includegraphics{heart_files/figure-latex/unnamed-chunk-10-1.pdf}

Environ 500 observations sont asymptomatiques, ce qui signifie que la
moitié (54\%) des individus ne présentaient aucun symptôme préalable
avant de subir une crise cardiaque.

\hypertarget{la-variable-restingbp}{%
\subsubsection{\texorpdfstring{La variable
\textbf{RestingBP}}{La variable RestingBP}}\label{la-variable-restingbp}}

\begin{Shaded}
\begin{Highlighting}[]
\FunctionTok{mean}\NormalTok{(RestingBP)}
\end{Highlighting}
\end{Shaded}

\begin{verbatim}
## [1] 132.3965
\end{verbatim}

\begin{Shaded}
\begin{Highlighting}[]
\FunctionTok{sd}\NormalTok{(RestingBP)}
\end{Highlighting}
\end{Shaded}

\begin{verbatim}
## [1] 18.51415
\end{verbatim}

\begin{Shaded}
\begin{Highlighting}[]
\FunctionTok{summary}\NormalTok{(RestingBP)}
\end{Highlighting}
\end{Shaded}

\begin{verbatim}
##    Min. 1st Qu.  Median    Mean 3rd Qu.    Max. 
##     0.0   120.0   130.0   132.4   140.0   200.0
\end{verbatim}

\begin{Shaded}
\begin{Highlighting}[]
\FunctionTok{boxplot}\NormalTok{(RestingBP)}
\end{Highlighting}
\end{Shaded}

\includegraphics{heart_files/figure-latex/unnamed-chunk-12-1.pdf}

\begin{Shaded}
\begin{Highlighting}[]
\FunctionTok{hist}\NormalTok{(RestingBP)}
\end{Highlighting}
\end{Shaded}

\includegraphics{heart_files/figure-latex/unnamed-chunk-12-2.pdf}

\begin{Shaded}
\begin{Highlighting}[]
\FunctionTok{qqnorm}\NormalTok{(RestingBP)}
\FunctionTok{abline}\NormalTok{(}\FunctionTok{mean}\NormalTok{(RestingBP),}\FunctionTok{sd}\NormalTok{(RestingBP),}\AttributeTok{col=}\DecValTok{2}\NormalTok{)}
\end{Highlighting}
\end{Shaded}

\includegraphics{heart_files/figure-latex/unnamed-chunk-12-3.pdf}

Le boxplot ainsi que l'histogramme nous montre qu'il ya des valeurs
abérantes. Le Q-Q plot ne suit pas très bien la ligne droite diagonale.
Il faudra donc pense à enlever les valeurs aberantes pour faire en sorte
que la variable suive une loi normale.

\hypertarget{la-variable-maxhr}{%
\paragraph{\texorpdfstring{La variable
\textbf{MaxHR}}{La variable MaxHR}}\label{la-variable-maxhr}}

\begin{Shaded}
\begin{Highlighting}[]
\FunctionTok{mean}\NormalTok{(MaxHR)}
\FunctionTok{sd}\NormalTok{(MaxHR)}
\FunctionTok{summary}\NormalTok{(MaxHR)}
\end{Highlighting}
\end{Shaded}

\begin{verbatim}
## [1] 136.8094
## [1] 25.46033
##    Min. 1st Qu.  Median    Mean 3rd Qu.    Max. 
##    60.0   120.0   138.0   136.8   156.0   202.0
\end{verbatim}

\begin{Shaded}
\begin{Highlighting}[]
\FunctionTok{boxplot}\NormalTok{(MaxHR)}
\end{Highlighting}
\end{Shaded}

\includegraphics{heart_files/figure-latex/unnamed-chunk-14-1.pdf}

\begin{Shaded}
\begin{Highlighting}[]
\FunctionTok{hist}\NormalTok{(MaxHR)}
\end{Highlighting}
\end{Shaded}

\includegraphics{heart_files/figure-latex/unnamed-chunk-14-2.pdf}

\begin{Shaded}
\begin{Highlighting}[]
\FunctionTok{qqnorm}\NormalTok{(MaxHR)}
\FunctionTok{abline}\NormalTok{(}\FunctionTok{mean}\NormalTok{(MaxHR),}\FunctionTok{sd}\NormalTok{(MaxHR),}\AttributeTok{col=}\DecValTok{2}\NormalTok{)}
\end{Highlighting}
\end{Shaded}

\includegraphics{heart_files/figure-latex/unnamed-chunk-14-3.pdf}

Le boxplot montre que la variable prend beaucoup de modalités allant de
60 jusqu'à 202. Cependant, le boxplot est assez symétrique (n'est pas
dispersé en haut ni en bas) avec une moyenne (136.8) proche de la
médiane (138).

Avec l'histogramme on voit que le mode est à 120 - 150 et que la
répartition est assez symétrique. Le Q-Q plot suit en grande partie la
ligne droite diagonale. On peut donc supposer que \textbf{MaxHR} suit
une loi normale.

\hypertarget{les-estimateurs-de-moyenne-et-variance}{%
\subsection{Les estimateurs de moyenne et
variance}\label{les-estimateurs-de-moyenne-et-variance}}

\hypertarget{intervalles-de-confiance-des-moyennes}{%
\subsection{Intervalles de confiance des
moyennes}\label{intervalles-de-confiance-des-moyennes}}

\hypertarget{intervalle-de-confiance-de-la-variable-cholesterol}{%
\paragraph{\texorpdfstring{Intervalle de confiance de la variable
\textbf{Cholesterol}}{Intervalle de confiance de la variable Cholesterol}}\label{intervalle-de-confiance-de-la-variable-cholesterol}}

La valeur normale de cholestérol total varie en fonction de l'âge et du
sexe. Chez une femme de 45 à 60 ans, il doit être compris entre 155
mg/dL et 255 mg/dL. Au-delà de 60 ans, il doit être compris entre 140 et
265 mg/dL.

On se pose la question à savoir si le cholesterol moyen est
significativement différent de 230 mg/dL ?

\begin{Shaded}
\begin{Highlighting}[]
\NormalTok{cholesterol }\OtherTok{\textless{}{-}}\NormalTok{ Cholesterol[Cholesterol}\SpecialCharTok{!=}\DecValTok{0}\NormalTok{]}
\NormalTok{interval\_chol }\OtherTok{\textless{}{-}} \FunctionTok{t.test}\NormalTok{(cholesterol,}\AttributeTok{mu =} \DecValTok{230}\NormalTok{)}
\end{Highlighting}
\end{Shaded}

\textbf{L'Hypothèse H0 est: la moyenne m = 230}

Dans le résultat ci-dessus : t est la statistique de Student (t =
6.7576), \textbf{df} est le degré de liberté (df= 745), \textbf{p-value}
est le degré de significativité du test (p-value = 2.83 × 10-11).
L'intervalle de confiance de la moyenne à 95\% est également montrée
(intervalle de confiance= {[}240.3837 248.8871{]}); et enfin, on a la
valeur moyenne de la série x (moyenne = 244.6354).

La \textbf{p-value} du test est de 2.83 × 10-11. Ce qui est largement
inférieur à 0.05. On rejette l'hypothèse 0 et on conclut que le
cholesterol moyen des souris est significativement différent de 200g
avec une \textbf{p-value} = 2.83 × 10-11.

\hypertarget{intervalle-de-confiance-de-la-variable-cholesterol-1}{%
\paragraph{\texorpdfstring{Intervalle de confiance de la variable
\textbf{Cholesterol}}{Intervalle de confiance de la variable Cholesterol}}\label{intervalle-de-confiance-de-la-variable-cholesterol-1}}

\hypertarget{analyse-multivariuxe9e}{%
\section{Analyse multivariée}\label{analyse-multivariuxe9e}}

\hypertarget{quali-x-quali}{%
\subsection{Quali x Quali}\label{quali-x-quali}}

Interessons nous de plus près à l'age des individus.

Regardons la variable \textbf{Age} en fonction de la variable
\textbf{Sex}

\begin{Shaded}
\begin{Highlighting}[]
\NormalTok{age\_sex\_hist }\OtherTok{\textless{}{-}} \FunctionTok{ggplot}\NormalTok{(heart, }\FunctionTok{aes}\NormalTok{(}\AttributeTok{x=}\NormalTok{Age, }\AttributeTok{fill=}\NormalTok{Sex, }\AttributeTok{color=}\NormalTok{Sex)) }\SpecialCharTok{+}
  \FunctionTok{geom\_histogram}\NormalTok{(}\AttributeTok{binwidth=}\DecValTok{1}\NormalTok{,}\AttributeTok{position=}\StringTok{"identity"}\NormalTok{, }\AttributeTok{alpha=}\FloatTok{0.5}\NormalTok{) }\SpecialCharTok{+}
  \FunctionTok{geom\_vline}\NormalTok{(}\AttributeTok{data=}\NormalTok{heart, }\FunctionTok{aes}\NormalTok{(}\AttributeTok{xintercept=}\FunctionTok{mean}\NormalTok{(Age), }\AttributeTok{color=}\NormalTok{Sex), }\AttributeTok{linetype=}\StringTok{"dashed"}\NormalTok{) }\SpecialCharTok{+}
  \FunctionTok{labs}\NormalTok{(}\AttributeTok{title=}\StringTok{"Distribution of age by sex"}\NormalTok{) }\SpecialCharTok{+}
  \FunctionTok{facet\_grid}\NormalTok{(}\SpecialCharTok{\textasciitilde{}}\NormalTok{Sex)}

\FunctionTok{print}\NormalTok{(age\_sex\_hist)}
\end{Highlighting}
\end{Shaded}

\includegraphics{heart_files/figure-latex/unnamed-chunk-16-1.pdf}

Les observations masculines sont plus denses entre 50 et 65 ans.

On peut vérifier cela en regardant l'effectif d'hommes et femmes

\begin{Shaded}
\begin{Highlighting}[]
\FunctionTok{table}\NormalTok{(Sex)}
\end{Highlighting}
\end{Shaded}

\begin{verbatim}
## Sex
##   F   M 
## 193 725
\end{verbatim}

On voit qu'il y'a 3.7x plus d'hommes que de femmes dans cette étude.

On vérifie si la moyenne d'age pour les deux sex sont proche

\begin{Shaded}
\begin{Highlighting}[]
\NormalTok{heart }\SpecialCharTok{\%\textgreater{}\%}
  \FunctionTok{group\_by}\NormalTok{(Sex) }\SpecialCharTok{\%\textgreater{}\%}
  \FunctionTok{summarise}\NormalTok{(}\AttributeTok{mean =} \FunctionTok{mean}\NormalTok{(Age), }\AttributeTok{n =} \FunctionTok{n}\NormalTok{())}
\end{Highlighting}
\end{Shaded}

\begin{verbatim}
## # A tibble: 2 x 3
##   Sex    mean     n
##   <chr> <dbl> <int>
## 1 F      52.5   193
## 2 M      53.8   725
\end{verbatim}

Les hommes ont une moyenne d'age de 53.8 ans.

Les femmes ont une moyenne d'age de 52.5 ans.

Bien qu'il y ait 3.7x plus d'hommes que de femmes, les deux moyennes
sont assez proche. La variable age ne va donc pas biaiser l'analyse en
faveur d'un des sex.

On s'interesse maintenant à voir les maladies cardiaque en fonction de
l'age.

\begin{Shaded}
\begin{Highlighting}[]
\FunctionTok{ggplot}\NormalTok{(heart,}\FunctionTok{aes}\NormalTok{(}\AttributeTok{x=}\NormalTok{Age,}\AttributeTok{fill=}\NormalTok{HeartDisease,}\AttributeTok{color=}\NormalTok{HeartDisease)) }\SpecialCharTok{+}
\FunctionTok{geom\_histogram}\NormalTok{(}\AttributeTok{binwidth =} \DecValTok{1}\NormalTok{,}\AttributeTok{color=}\StringTok{"black"}\NormalTok{)}\SpecialCharTok{+}
\FunctionTok{stat\_count}\NormalTok{(}\AttributeTok{geom =} \StringTok{"text"}\NormalTok{, }\FunctionTok{aes}\NormalTok{(}\AttributeTok{label =} \FunctionTok{stat}\NormalTok{(count)), }
             \AttributeTok{position =} \FunctionTok{position\_stack}\NormalTok{(), }\AttributeTok{color =} \StringTok{"black"}\NormalTok{)}\SpecialCharTok{+}
\FunctionTok{ylab}\NormalTok{(}\StringTok{"Effectif"}\NormalTok{)}
\end{Highlighting}
\end{Shaded}

\includegraphics{heart_files/figure-latex/unnamed-chunk-19-1.pdf}

Sur ce graphique, on peut voir le nombre de personne qui ont des
maladies cardiaque en fonction de l'age. Entre 54 et 63 ans le nombre de
maladies cardiaque est le plus élevé.

Cependant, à première vue, ce graphique peut induire en erreur dans le
sens où si on regarde pour l'age 72, on voit qu'il n'ya que 3 maladies
cardiaque comparé 28 maladies cardiaque pour un age de 58 an. Ceci est
du au fait qu'il ya beacoup moins de données pour l'age 72 (4 données)
contrairement à l'age 58 (42 données)

Il faut donc aussi voir par proportion d'age.

\begin{Shaded}
\begin{Highlighting}[]
\FunctionTok{ggplot}\NormalTok{(heart,}\FunctionTok{aes}\NormalTok{(}\AttributeTok{x=}\NormalTok{Age,}\AttributeTok{fill=}\NormalTok{HeartDisease,}\AttributeTok{color=}\NormalTok{HeartDisease)) }\SpecialCharTok{+}
\FunctionTok{geom\_histogram}\NormalTok{(}\AttributeTok{position =} \StringTok{"fill"}\NormalTok{, }\AttributeTok{binwidth =} \DecValTok{1}\NormalTok{, }\AttributeTok{alpha=}\FloatTok{0.7}\NormalTok{)}\SpecialCharTok{+}
\FunctionTok{ylab}\NormalTok{(}\StringTok{"Proportion de maladie cardiaque"}\NormalTok{) }\SpecialCharTok{+}
\FunctionTok{ggtitle}\NormalTok{(}\StringTok{"Proportion de maladie cardiaque en fonction de l\textquotesingle{}age"}\NormalTok{)}
\end{Highlighting}
\end{Shaded}

\includegraphics{heart_files/figure-latex/unnamed-chunk-20-1.pdf}

Ici, on voit bien que plus l'age augmente, plus la proportion de
maladies cardiaque augmente. On voit par exemple que 75\% des personne
ayant 72 ans, ont des maladies cardiaque contre 64\% pour ceux ayant 58
ans.

Regardons maintenant la variable \textbf{ChestPainType} en fonction de
la variable \textbf{Sex}

\begin{Shaded}
\begin{Highlighting}[]
\NormalTok{chest\_pain\_bar }\OtherTok{\textless{}{-}} \FunctionTok{ggplot}\NormalTok{(heart, }\FunctionTok{aes}\NormalTok{(}\AttributeTok{x=}\NormalTok{ChestPainType, }\AttributeTok{fill=}\NormalTok{ChestPainType)) }\SpecialCharTok{+}
  \FunctionTok{geom\_bar}\NormalTok{() }\SpecialCharTok{+}
  \FunctionTok{facet\_grid}\NormalTok{(}\SpecialCharTok{\textasciitilde{}}\NormalTok{Sex) }\SpecialCharTok{+}
  \FunctionTok{labs}\NormalTok{(}\AttributeTok{title=}\StringTok{"Distribution of chest pain type"}\NormalTok{)}

\FunctionTok{print}\NormalTok{(chest\_pain\_bar)}
\end{Highlighting}
\end{Shaded}

\includegraphics{heart_files/figure-latex/unnamed-chunk-21-1.pdf}

\end{document}
